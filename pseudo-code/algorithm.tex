% -- coding: UTF-8 --
\documentclass{article}

% 导言区:调用宏包、进行文档的全局设置
\usepackage[UTF8]{ctex}
\usepackage[ruled, linesnumbered]{algorithm2e}

% 算法部分
\SetKwInOut{KwIn}{输入}
\SetKwInOut{KwOut}{输出}
\SetNlSty{}{}{:}

% 正文区:书写文档的正式内容
\begin{document}
	
	% 算法1
	\begin{algorithm}
		\renewcommand{\thealgocf}{3-1}
		\renewcommand{\algorithmcfname}{降维算法}
		\caption{PCA(主成份分析)}
		\KwIn{样本集$ D=\left\{ x_{1},x_{2},\ldots,x_{m} \right\} $;\\低维空间维数$ d^{'} $.}
		\KwOut{投影矩阵$ \textbf{W}=\left( w_{1},w_{2},\ldots,w_{d^{'}} \right) $.}
		对所有样本进行中心化:$ x_{i}\leftarrow x_{i}-\frac{1}{m}\sum_{i=1}^{m}x_{i} $\;
		计算样本的协方差矩阵$ \textbf{X}\textbf{X}^{T} $\;
		对协方差矩阵$ \textbf{X}\textbf{X}^{T} $做特征值分解\;
		取最大的$ d^{'} $个特征值所对应的特征向量$ w_{1},w_{2},\ldots,w_{d^{'}} $.
	\end{algorithm}

	% 算法2
	\begin{algorithm}
		\renewcommand{\thealgocf}{4-1}
		\renewcommand{\algorithmcfname}{聚类算法}
		\caption{K-means}
	\end{algorithm}

\end{document}
